\documentclass[letterpaper]{article}
\usepackage[spanish,es-tabla]{babel}
\usepackage{indentfirst}
\usepackage{float}
\usepackage[utf8]{inputenc}
\usepackage{graphicx}

\begin{document}
    
Aplicando la eq.\ref{eq.DFT} se obtiene 

\begin{equation}
    x[k] = \sum_{n=0}^{7} \sin \left( \frac{\pi }{4} n \right) e^{-j\frac{\pi }{4}nk} 
\end{equation}

Para desarrollar la sumatoria de forma mas sencilla la sumatoria presentan los valores de $x[n]$ para $n \in \{0, 1, 2, 3, 4 , 5, 6, 7\}$ en la 
Tabla.\ref{tab.xn}

\begin{table}[H]
    \centering
    \begin{tabular}{|c|c|c|}
        \hline $n$ &  $x[n]$\\ 
        \hline $0$ & $0$ \\
        \hline $1$ & $\sqrt{2}/2$ \\
        \hline $2$ & $1$ \\
        \hline $3$ & $\sqrt{2}/2$ \\
        \hline $4$ & $0$ \\
        \hline $5$ & $-\sqrt{2}/2$ \\
        \hline $6$ & $-1$ \\
        \hline $7$ & $-\sqrt{2}/2$ \\
        \hline
    \end{tabular}
    \caption{Valores de $x[n]$ para un periodo.}
    \label{tab.xn}

\end{table}

Con lo que se obtiene 

\begin{equation}
    X[k]= \frac{\sqrt{2}}{2} e^{-j\frac{\pi }{4}k} + e^{-j\frac{\pi }{2}k} + \frac{\sqrt{2}}{2} e^{-j\frac{3\pi }{4}k}
    - \frac{\sqrt{2}}{2} e^{-j\frac{5\pi }{4}k} - e^{-j\frac{3\pi }{2}k} - \frac{\sqrt{2}}{2} e^{-j\frac{7\pi }{4}k} 
\end{equation}

Si expresamos las exponenciales correspondientes para $n \in \{ 5, 6 ,7 \}$ como $m\pi/4 + \pi$ donde $m$ es un entero entre 
$0$ y $4$, es posible expresar 

\begin{equation}
    X[k] = \frac{\sqrt{2}}{2} e^{-j\frac{\pi }{4}k} + e^{-j\frac{\pi }{2}k} + \frac{\sqrt{2}}{2} e^{-j\frac{3\pi }{4}k}
    - \frac{\sqrt{2}}{2} e^{-j\frac{\pi }{4}k} e^{-j\pi k} - e^{-j\frac{\pi }{2}k}e^{-j\pi k} - \frac{\sqrt{2}}{2} e^{-j\frac{3\pi }{4}k} e^{-j\pi k}
\end{equation}

Como $k$ es entero y $e^{-j\pi k}$ es siempre multiplo de $\pi$ entonces $e^{-j\pi k} = (-1)^k$.

\begin{equation}
    X[k] = \frac{\sqrt{2}}{2} e^{-j\frac{\pi }{4}k} + e^{-j\frac{\pi }{2}k} + \frac{\sqrt{2}}{2} e^{-j\frac{3\pi }{4}k}
    (-1)^{1+k} \frac{\sqrt{2}}{2} e^{-j\frac{\pi }{4}k} +(-1)^{1+k} e^{-j\frac{\pi }{2}k} + (-1)^{1+k}\frac{\sqrt{2}}{2} e^{-j\frac{3\pi }{4}k} 
\end{equation}

Agrupando los valores se obtiene 

\begin{equation}
    X[k]= [ 1 + (-1)^{1+k} ] ( \frac{\sqrt{2}}{2} e^{-j\frac{\pi }{4}k} +e^{-j\frac{\pi }{2}k} + 
    \frac{\sqrt{2}}{2}  e^{-j\frac{3\pi }{4}k} )
    \label{eq.xk}
\end{equation}


Sabiendo que $X[k]$ es periodica con periodo $N$, en este caso $8$, se muentran en tab.\ref{tab.xk} los valores obtenidos para 
los diferentes valores de $k$ en un periodo, por otro lado de la eq.\ref{eq.xk} se sabe que si $k$ es par $X[k]$ es $0$.

\begin{table}[H]
    \centering
    \begin{tabular}{|c|c|}
        \hline $k$ & $X[k]$ \\ 
        \hline $0$ & $0$ \\ 
        \hline $1$ & $-4j$ \\ 
        \hline $2$ & $0$ \\ 
        \hline $3$ & $0$ \\ 
        \hline $4$ & $0$ \\ 
        \hline $5$ & $0$ \\ 
        \hline $6$ & $0$ \\ 
        \hline $7$ & $4j$ \\ 
        \hline 
        
    \end{tabular}
    \caption{$X[k]$ obtenidos para un periodo.}
    \label{tab.xk}

\end{table}


\end{document}