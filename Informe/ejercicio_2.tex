\documentclass[letterpaper]{article}
\usepackage[spanish,es-tabla]{babel}
\usepackage{indentfirst}
\usepackage{float}
\usepackage[utf8]{inputenc}
\usepackage{graphicx}

\title{Evaluación DSP}
\author{Saez, Lautaro Andres}

\begin{document}
    \maketitle

    \section{Marco Teorico}

    \subsection{DFT}

    Se define la \textbf{DFT} de una secuencia $x[n]$ de $N$ muestras como

    \begin{equation}
        \label{eq.DFT}
        X[k]=\sum_{n=0}^{N-1} x[n]e^{-j\frac{2\pi }{N} nk }
    \end{equation}

    \subsection{CTFS}

    La \textbf{CTFS} de una señal podemos calcularla como 

    \begin{equation}
        a_k = \
    \end{equation}

    \section*{Ejercicio 2}

    Se utilizara la siguiente señal 

    \begin{equation}
        \label{eq.punto2}
        x_c(t) = sin( 200 \pi t )
    \end{equation}

    \subsection*{Calculo de la CTFS}

    Aplicando la identidad de Euler a la eq.\ref{eq.punto2} se obtiene 

    \begin{equation}
        x_c(t)=\frac{1}{2j}( e^{j200\pi t} - e^{-j200 \pi t})
    \end{equation}

    Es posible apreciar que los coeficientes de la serie son: 

    \begin{equation}
        c_1 = \frac{1}{2j} \hspace*{.1cm} \land \hspace*{.1cm} c_{-1}=-\frac{1}{2j}
    \end{equation}

    \subsection*{Muestreo}

    Aplicando el teorema del muestreo, se debe elegir una frecuencia mayor a $2F_N$, siendo 
    en este caso $F_N=100Hz$, por ello se utilizara $F_s=400Hz$. Con dicha frecuencia se obtiene 

    \begin{equation}
        x[n]=x(n/F_s) \rightarrow x[n]= sin\left( \frac{\pi}{2} n\right)
    \end{equation}

    Cuyo periodo fundamental ($N_0$) es $4$.

    Aplicando la eq.\ref{eq.DFT}

    \begin{equation}
        X[k]= \sum_{n=0}^{3} sin\left( \frac{\pi}{2} n\right) e^{-j\frac{\pi}{2} nk}
    \end{equation}

    Calculando los valores de $x[n]$ para $n\in {1, 2, 3 ,4}$

    \begin{table}[H]
        \centering
        \begin{tabular}{|c|c|}
            \hline $n$ & $sin(\pi n/2)$ \\ 
            \hline $0$ & $0$ \\
            \hline $1$ & $1$ \\
            \hline $2$ & $0$ \\
            \hline $3$ & $-1$ \\
            \hline
        \end{tabular}
        \caption{Valores de la señal muestreada en 1 periodo.}
    \end{table}

    Reemplazando en la ecuacion anterior, se obtiene 

    \begin{equation}
        X[k]= e^{-j\frac{\pi}{2}k} - e^{-j\frac{ 3 \pi}{2}k}
    \end{equation}

    Es posible expresar a $e^{-j\frac{ 3 \pi}{2}k}$ como $e^{-j \frac{\pi }{2}k}e^{-j\pi k}$, aplicando la identidad de Euler 
    $e^{-j \pi k}= (-1)^k$. 

    \begin{equation}
        X[k]= [ 1 + (-1)^{k+1} ] e^{ -j \frac{\pi }{2} k }
    \end{equation}

    \begin{table}[H]
        \centering
        \begin{tabular}{|c|c|}
            \hline $k$ & $X[k]$ \\
            \hline $0$ & $0$ \\
            \hline $1$ & $-2j$ \\
            \hline $2$ & $0$ \\
            \hline $3$ & $2j$ \\
            \hline
        \end{tabular}
        \caption{Valores de $X[k]$.}
        \label{tab.xk}
    \end{table}

    \subsection{CTFS a partir de la DFT}

    Al dividir los coeficientes $X[k]$ (Tab.\ref{tab.xk}) por el el numero de muestras $N$, el resultado son los coeficientes 
    $c_k$. Luego se puede establecer a priopi la relacion 
    
    \begin{equation}
        c_k = \frac{X[k]}{N}
    \end{equation}

    Para la ecuacion anterior se aplica el conocimiento que $X[k]$ es periodica, y por lo tanto $X[-1]=X[3]$.

\end{document}